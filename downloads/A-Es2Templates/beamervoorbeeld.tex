\documentclass[usepdftitle,invisible]{beamer}
\usetheme{aes2}
\usepackage{aes,pdfpages}
\begin{document}
\author{\aeskwadraat \TeXniCie\\
 \url{hektex@a-eskwadraat.nl}}
\title{De beamer class}
\date{\today}
\begin{frame}
\titlepage
\end{frame}
\begin{frame}
 \frametitle{Inhoudsopgave}
 \tableofcontents
\end{frame}
\section{Eerste onderwerp}
\subsection{Eerste deelonderwerp}
\begin{frame}{Een kolom, omdat het kan}
     \begin{columns}[T]
     \begin{column}[T]{5cm}
     Niet alles hoef altijd nuttig te zijn, soms doe je dingen omdat ze mooi zijn \\
     \end{column}
     \begin{column}[T]{5cm}
     \begin{figure}
          \includegraphics[height=3cm]{Mandelbrot_set_image}
     \end{figure}

     \end{column}
     \end{columns}
\end{frame}
\subsubsection{Eerste onderdeel}
\begin{frame}
\frametitle{Zomaar een slide}
\begin{itemize}
\item Presentaties maken in \LaTeX\ is simpel.
\item Formules zijn eenvoudig in te voegen:
\end{itemize}
$$\int^\infty_0 e^{-x^2} \text{d}x = \frac{\sqrt{\pi}}{2}$$
En ja de Gaussian komt inderdaad erg veel voor vandaag.
\end{frame}

\section{Tweede deelonderwerp}
\subsection{Eerste onderdeel}
\begin{frame}{Een animatie}
  \begin{itemize}[<+->]
     \item Veel serieuzer wordt het er zo niet op
     \item KiekeBoe!
  \end{itemize}
\end{frame}
\end{document}
