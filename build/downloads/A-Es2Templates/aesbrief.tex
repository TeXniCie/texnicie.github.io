% rubber: depend aesbrief-voorbeeld.tex
\documentclass{article}

\usepackage[dutch]{babel}
\usepackage{url,hyperref,enumitem}
\usepackage{a4wide}
\usepackage{aes,cursus,pdfpages,listings}

\setlength\parindent{0cm}
\setlength\parskip{1ex}

\setdescription{style=nextline}

\newcommand\meta[1]{\placeholder[#1]}
\newcommand\marg[1]{\cmdarg{\meta{#1}}}

\newcommand\classnaam{aesbrief}% Hier de naam van de class van de handleiding
\newcommand\classsf{\textsf{\classnaam}}
\newcommand\funktie{\aesnaam brief}% Hier de functie van de class (brief, notulen, ed)

\title{De \classsf-class} %\\{\large versie \fileversion}}
\author{\aeskwadraat \TeXniCie\\
\url{hektex@a-eskwadraat.nl}}
%\date{\filedate}

\begin{document}

\maketitle


\section{Introductie}%Korte uitleg van het doel van het Package, voorbeeld van aesbrief

De \classsf-class vormt de standaard \funktie \ van \aeskwadraat.
Dit document legt uit hoe je een \funktie \ brief maakt en hoe de verschillende commando's werken.


\section{De class laden}

Met \cmd{documentclass}\cmdarg[opt]{\meta{opties}}\cmdarg{\classnaam} bovenaan je document laad je de \classsf-class.
De \meta{opties} (gescheiden door komma's) zijn:

\begin{description}
\item[Opties voor \classnaam] 
debug, nodebug, kleur, briefpapier, nokleur, geenkleur, footer, nofooter, ontour, betadag en alle opties van de aestaal package.
\item[Opmerking] De optie kleur is momenteel noodzakelijk wegens onderhoudt.
\end{description}

Overige opties worden doorgegeven aan de article-class.

\section{Informatie opgeven} % Uitleg over de specifieke informatie die opgegeven moet worden voor de class

Er zijn allerlei dingen die je kunt of moet instellen. Dat gebeurt door middel
van allerlei commando's die je vrijwel overal tussen \class{\classnaam} 
en \cmd{opening} (zie sectie \ref{sec:handleiding}) kunt plaatsen.

\subsection{Verplicht}

Sommige commando's zijn `verplicht', als je ze weglaat
zal \classsf{} klagen, maar zijn best doen om toch een \funktie \ te produceren.

\begin{description}
\item[\cmd{opening}\marg{aanhef}] Hiermee geef je aan waarmee je de brief wilt beginnen.
\end{description}

\begin{description}
	\item[\cmd{begin\{brief\}}\marg{gegevens ontvanger}] Achter de \cmd{begin\{brief\}} zet je de gegevens van de ontvanger zoals naam, adres etc.
\end{description}

\begin{description}
	\item[\cmd{closing}\marg{sluiting}] Hier geef je aan waarmee je de brief wilt afsluiten, zoals ''Met vriendelijke groet''.
\end{description}

\begin{description}
	\item[\cmd{signature}\marg{Naam verzender}] Hier geef je de naam van de verzender aan.
\end{description}

\begin{description}
	\item[\cmd{email}\marg{mail-adres}] Geef hier je mail-adres op.
\end{description}
%
%\begin{description}
%	\item[\cmd{commando}\marg{eis}] Uitleg over commando en eis.
%\end{description}


\subsection{Optioneel}

De volgende commando's kun je gebruiken om optionele informatie op te geven.

\begin{description}
\item[\cmd{subject}\marg{onderwerp}] Hiermee kan je je brief een onderwerp geven.
\end{description}

\begin{description}
	\item[\cmd{bijlagen}\marg{\cmd{item} bijlage 1 \cmd{item} bijlage 2 ...}] Hiermee kan je bijlages aangeven.
\end{description}

\begin{description}
	\item[\cmd{uwk}\marg{kenmerk ontvanger}] De mogelijkheid om een kenmerk van de ontvanger toe te voegen.
\end{description}

\begin{description}
	\item[\cmd{onsk}\marg{kenmerk verzender}] De mogelijkheid om een kenmerk van de verzender toe te voegen.
\end{description}

\begin{description}
	\item[\cmd{cienaam}\marg{commissie naam}]  De mogelijkheid om een commisCie aan te geven.
\end{description}

\begin{description}
	\item[\cmd{datum}\marg{datum}] Uitleg over commando en eis.
\end{description}

%\begin{description}
%	\item[\cmd{commando}\marg{eis}] Uitleg over commando en eis.
%\end{description}
%
%\begin{description}
%	\item[\cmd{commando}\marg{eis}] Uitleg over commando en eis.
%\end{description}
%
%\begin{description}
%	\item[\cmd{commando}\marg{eis}] Uitleg over commando en eis.
%\end{description}

\section{De handleiding zelf}\label{sec:handleiding}

De aesbrief heeft het format van een normale brief. Het grootste verschil is dat hier al je gegevens van a-eskwadraat al zijn ingevuld. Verder zijn de commando's lekker duidelijk.

\section{Voorbeeld}

We sluiten af met een voorbeeld.
In figuur \ref{fig:code} zie je een voorbeeld-\LaTeX-bestand. De resulterende \funktie \ zie je in figuur \ref{fig:brief}.

\begin{figure}[ht]
\centering
\fbox{%
\begin{minipage}{.9\textwidth}
\verbatiminput{\classnaam voorbeeld.tex}
\end{minipage}%
}
\caption{Een voorbeeld van het gebruik van \classsf.}
\label{fig:code}
\end{figure}

\begin{figure}\begin{center}
\fbox{\includegraphics[width=.9\textwidth]{\classnaam voorbeeld.pdf}}
\caption{De gezette \funktie \ die het resultaat is van de code in figuur \ref{fig:code}.}
\label{fig:brief}\end{center}
\end{figure}


\end{document}
