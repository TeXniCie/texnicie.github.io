% rubber: depend aesbrief-voorbeeld.tex
\documentclass[a4paper]{article}

\usepackage[dutch]{babel}
\usepackage{url,hyperref,enumitem}
\usepackage{a4wide}
\usepackage{pdfpages}
\usepackage{aes,cursus,pdfpages}

\setlength\parindent{0cm}
\setlength\parskip{1ex}

\setdescription{style=nextline}

\newcommand\meta[1]{\placeholder[#1]}
\newcommand\marg[1]{\cmdarg{\meta{#1}}}

\newcommand\classnaam{contract}% Hier de naam van de class van de handleiding TEST
\newcommand\classsf{\textsf{\classnaam}}
\newcommand\funktie{contract}% Hier de functie van de class (brief, notulen, ed)

\title{De \classsf-class} %\\{\large versie \fileversion}}
\author{\aeskwadraat \TeXniCie\\
\url{hektex@a-eskwadraat.nl}}
%\date{\filedate}

\begin{document}

\maketitle


\section{Introductie}%Korte uitleg van het doel van het Package, voorbeeld van aesbrief

De \classsf-class vormt een standaard \funktie \ van \aeskwadraat.
Dit document legt uit hoe je een \funktie maakt en hoe de verschillende commando's werken.


\section{De class laden}

Met \cmd{documentclass}\cmdarg[opt]{\meta{opties}}\cmdarg{\classnaam} bovenaan je document laad je de \classsf-class.
De \meta{opties} (gescheiden door komma's) zijn:

\begin{description}
\item[english] Zorgt dat het contract in het Engels is.
\end{description}

Overige opties worden doorgegeven aan de article-class.

\section{Informatie opgeven} % Uitleg over de specifieke informatie die opgegeven moet worden voor de class

Er zijn allerlei dingen die je kunt of moet instellen. Dat gebeurt door middel
van allerlei commando's die je vrijwel overal tussen \class{\classnaam} 
en \cmd{begin\cmdarg{document}} (zie sectie \ref{sec:handleiding}) kunt plaatsen.

\subsection{Verplicht}

Sommige commando's zijn `verplicht', als je ze weglaat
zal \classsf{} klagen, maar zijn best doen om toch een \funktie \ te produceren.

\begin{description}
\item[\cmd{settegenpartij}\marg{bedrijf}] Vul hier de naam van de tegenpartij, met het volledige adres.
\item[\cmd{setTegenpartijkort}\marg{korte naam Bedrijf}] Stel hier de korte naam van de tegenpartij in (wordt gebruikt aan het begin van regels).
\item[\cmd{settegenpartijkort}\marg{korte naam bedrijf}] Stel hier de korte naam van de tegenpartij in (wordt gebruikt midden in de zin)
\end{description}

\subsection{Optioneel}

De volgende commando's kun je gebruiken om optionele informatie op te geven.

\begin{description}
\item[\cmd{welbtw}] Met dit commando wordt er automatisch BTW bijgeteld bij de benoemde bedragen, en wordt de BTW-clausule ook opgenomen.
\item[\cmd{annulering}] Hiermee worden er annuleringsvoorwaarden toegevoegd aan het contract. Plaats deze clausule aan het einde van je document, net v\'o\'or \cmd{end\cmdarg{document}}
\end{description}

\section{Het contract zelf}\label{sec:handleiding}

Als je hebt gedefinieerd wat de tegenpartij is, kun je de inhoud zelf van contract toevoegen. Dit doe je tussen \cmd{begin\cmdarg{document}} en \cmd{end\cmdarg{document}}.
Een item voor je contract (bijvoorbeeld het leveren van Goodies, het plaatsen van een logo op promotiemateriaal, ...) voeg je toe met een \cmd{section}.\\

Gebruik voor de naam van de sponsor het commando \cmd{spons}, en voor de naam van \aesnaam het commando \cmd{aes}.\\

Sluit deze sectie af met het gevraagde bedrag (standaard exclusief BTW) met het commando \cmd{bedrag\cmdarg{bedrag in euro}}.
Als je meerdere items in je contract zet, dan worden de bedragen automatisch opgeteld.

\section{Voorbeeld}

We sluiten af met een voorbeeld.
In sectie \ref{sec:code} zie je een voorbeeld-\LaTeX-bestand. De resulterende \funktie \ zie je in sectie \ref{sec:voorbeeld}.

\clearpage
\subsection{Voorbeeld code}\label{sec:code}
\begin{figure}[ht]
\fbox{%
\begin{minipage}{.9\textwidth}
\verbatiminput{\classnaam voorbeeld.tex}
\end{minipage}%
}
\caption{Een voorbeeld van het gebruik van \classsf.}
\label{fig:code}
\end{figure}

\clearpage
\includepdf[scale=.7,clip,frame=true,pages={1},pagecommand={\subsection{Voorbeeld PDF}\label{sec:voorbeeld}}]{\classnaam voorbeeld.pdf} 
\includepdf[scale=.7,clip,frame=true,pages={2-},pagecommand={}]{\classnaam voorbeeld.pdf}


\end{document}
